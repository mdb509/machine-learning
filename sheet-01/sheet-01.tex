\documentclass[ngerman,12pt]{article}

% PACKAGES
\usepackage[ngerman]{babel}
\usepackage[utf8]{inputenc}
\usepackage{url}
\usepackage{a4}
\usepackage{amsfonts}
\usepackage[usenames,dvipsnames]{xcolor}
\usepackage{verbatim}
\usepackage{graphicx}
\usepackage[hidelinks]{hyperref}
\usepackage{listings}
\usepackage{booktabs}

% PAGE LAYOUT
\setlength{\textwidth}{174mm}
\setlength{\textheight}{220mm}
\setlength{\topmargin}{10mm}
\setlength{\oddsidemargin}{10mm}
\setlength{\hoffset}{0mm}
\setlength{\voffset}{0mm}
\setlength{\parindent}{0mm}
\setlength{\parskip}{2mm}
\pagestyle{empty}

% EXERCISE SHEET SPECIFIC MAKROS
\renewcommand{\title}[1]{\par\vspace{5mm}\centerline{\Large #1}}
\newcommand{\subtitle}[1]{\par\vspace{1mm}\centerline{\footnotesize #1}\par}
\newcommand{\exercise}[1]{\par\vspace{3mm}{\bf Aufgabe{\hspace{1mm}}#1}\hspace{1mm}}
\definecolor{dark-gray}{gray}{0.40}
\newcommand{\TODO}[1]{{\sf\color{blue} [#1]}}

% GENERALLY USEFUL MATH MAKROS
\def\ceil#1{{\left\lceil#1\right\rceil}}
\def\floor#1{{\left\lfloor#1\right\rfloor}}
\def\mod{\mbox{ mod }}
\def\div{\mbox{ div }}
\def\sm{\backslash}
\def\IN{\mathbb{N}}
\def\IZ{\mathbb{Z}}
\def\IR{\mathbb{R}}
\def\Oh#1{O\left(#1\right)}
\def\Om#1{\Omega\left(#1\right)}
\def\Th#1{\Theta\left(#1\right)}
\def\eps{\varepsilon}

\begin{document}

% COLORS
\definecolor{darkblue}{rgb}{0.0,0.0,0.6}
\definecolor{darkred}{rgb}{0.7,0.0,0.0}

% HEAD OF EXERCISE SHEETS
\setlength{\textwidth}{16.5cm}
\setlength{\textheight}{22cm}
\setlength{\topmargin}{0cm}
\setlength{\oddsidemargin}{-0.3cm}
\setlength{\evensidemargin}{0cm}
\vspace*{-20mm}

% Links der Name des Lehrstuhls.
\parbox{40mm}{
Algori Thmus\\[0mm]
RZ: at314\\[0mm]
Mn: 2718281\\[-5mm]} %\\[0.3mm]
% In der Mitte der Name der Veranstaltung.
\parbox{100mm}{\vspace*{1mm}\begin{center}\large\bf%
Algorithmen und Datenstrukturen\\[0mm]
\hspace{5mm} SS 2023\\[0mm]
{\footnotesize\rm \url{http://ad-wiki.informatik.uni-freiburg.de/teaching}}%
\end{center}}
\par\vspace{-5mm}
% Grauer Strich und rechts das Uni-Logo.
\definecolor{freiburg-gray}{rgb}{0.68,0.68,0.68}
\vspace*{2mm}
\raisebox{1.19cm}{%
\textcolor{freiburg-gray}{\rule{0.885\textwidth}{1.1mm}}}
\par\vspace*{-37.84mm}
\hspace*{\fill}\includegraphics[width=74pt,height=107pt]{Uni_Logo.png}
\par\vspace{-5mm}

\bigskip

\title{Abgabe für das Übungsblatt N}
\subtitle{9. Mai 2023}
\renewcommand{\baselinestretch}{1.1}\normalsize

\bigskip

\exercise{1}

Ich habe einen wundervollen Kuchen gefunden, aber leider war hier nicht genug Platz, um ihn hinzuschreiben.

\exercise{2}

Sei $\eps <0$. Der Rest ist analog zur Vorlesung.

\exercise{3}

Die konzentrierte Beinhaltung als Kernstück eines zukunftsweisenden Parteiprogramms.

\exercise{4}

Folgt unmittelbar aus Aufgaben 1 - 3.




\end{document}
